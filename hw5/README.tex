\documentclass{article}

\title{HW5-README}
\author{Paula Charles}
\date{November 2021}
\begin{document}

\maketitle
\noindent
In this problem, I am reading a maze and finding
 a solution to go through it, from entrance to 
exit, knowing that there is an entrance at the 
top row of the maze 
and an exit at the bottom of it.
 I design this solution using the right hand
 following: I start with my hand on the right wall
 and I follow the wall with my hand up until the 
exit.
\newline

\noindent
To do that, I designed a C++ file which tackles 
this issue in several steps:
\begin{enumerate}
    \item I first compute the walls into a static
 array: the value of this array are set to 0 for 
the size of the maze (after verifying that the 
size of the maze is smaller than the size of the 
static array. I switch these values to 1 for every
 coordinate of a wall.
    \item Knowing where the walls are, I start going
 through the maze: I find out where the entrance is.
 Going from there, I define a direction: it is the 
direction I am facing and it will enable me to know 
where my right is and thus to know how to move. This
 direction is updated every time I take a step.
    \item Once I know the direction I am facing, I 
try to go to my right. If I can't (because there is 
a wall), I try to go straight. If I can't again 
because of a wall, I try to go left. Finally, if 
I can't go left, I go backwards.
    \item I repeat this step of moving and modifying
 my direction up until I am in the last row: that is
 where I exit. This should enable me to follow the 
right hand solution.
\end{enumerate}
\newline

\noindent
This solution is valid only if the maze has the right 
format: there is only one entrance, the exit is on 
the last row (and not on the sides) and it is smaller
 than my static array (201x201, but can be modified in
 the code).
\newline

\noindent
In the python file used to check if our C++ solution 
is valid, I created several functions which each 
are looking for a reason that would invalidate 
my solution: if they return True, it means my solution 
is valid for that issue.
\begin{itemize}
    \item The first function checks if the solution 
is going through any wall. For that, I created 2 sets: 
one with the wall coordinates, one with the path coordinates
: if they intersect, my solution does not work.
    \item The second function checks if the path is 
only moving one step at a time by comparing a set of 
coordinates for the path with the following one.
    \item The third function is checking if the solution 
is staying inside the maze boundaries. It is not needed 
if the maze is defined properly.
    \item The last function is checking if we are entering
 in the first row and exiting in the last one.
\end{itemize}
\newline

\noindent
If every function states that the solution is correct, an 
output will appear on the screen claiming that this solution 
is correct.
\end{document}

