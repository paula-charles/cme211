\documentclass{article}
\usepackage{algorithm2e}
\setlength{\parindent}{0pt}
\setlength{\parskip}{0.5em}

\title{Project part 1 - Writeup}
\author{Paula Charles}
\date{November 2021}

\begin{document}

\maketitle

\section{Explanation of the CG Solver}

To implement the Conjugate gradient code, I used several 
functions to prevent redundancy. These functions can be found
 in the matvecops.cpp file and enable us to 
make basic vector / matrix calculations. They are of several types :

First, functions that take only vectors as inputs: the 
function "addition" returns the sum of 2 vectors; the 
function "substraction" returns the substraction of the 
first vector passed as a parameter by the second one; the
function "mult vect" multiplies the transpose of a vector 
by another vector: it returns a double. I chose to use that 
last function instead of a more precise function that 
takes only one input and returns the multiplication of 
the transpose of this vector by itself, because there is 
one instance in the CG solver that uses 2 different vectors. 
The last function that takes a vector as the input is 
calculation of the L2 norm. It returns a double.

Then, there are 2 hybrid functions: one takes a matrix and a
 vector as inputs and multiplies the matrix by the vector 
on the right. The matrix is in the CSR format. The other 
function takes a scalar and a vector as inputs and returns 
the multiplication of that vector by the scalar.

All these functions do not modify their inputs. They either 
return a double or a vector. They enable me to write the 
CG algorithm in a smoother way, without having to detail 
the calculations every time.

We then use this CG solver in the main() function: we 
extract a COO matrix from the input file, that we convert 
in place to a CSR thanks to the function COO2CSR(). We then 
initialize the vector of solutions to a vector of ones and 
the vector on the right hand side to a vector of zeros. 
We also define the threshold as the size of the matrix. 
It is then that the CG algorithm is used.

This CG algorithm iterates over a vector of residuals, r0. 
It is updating the vector of solutions and the vector 
of residuals until the latest is smaller than a given threshold. In 
that case, we estimate that we have found the 
solution to the linear equations. We then return the number 
of iterations it took to reach this threshold and update 
the vector of solutions in place.


\section{CG algorithm}

\begin{algorithm}[H]
 \SetAlgoLined
 \KwData{matrix A, vector b, vector x, double tol}
 \KwResult{Number of iterations to reach convergence (-1 if not convergent), updated vector of solutions}
 Initialization: u0 = x;
 
 success = 0; \tcp{it will tell us if algorithm converges}
 
 r0 = b - A * u0;
 
 L2norm\_r0 = L2norm(r0);
 
 p0 = r0;
 
 niter = 0; \tcp{number of iterations}
 
 niter\_max = number of rows;
 
 \While{niter < niter\_max}{
 niter = niter + 1;
 
 alpha = transp(r0)*r0 / ( transp(p0)*A*p0 );
 
 u0 = u0 + alpha*p0;
 
 r1 = r0 - alpha*A*p0;
 
 L2norm\_r1 = L2norm(r1);
  
 \eIf{L2norm\_r1/L2norm\_r0 < tol}{
 success = 1;
 
 break;}
 {beta = transp(r1)*r1 / transp(r0)*r0;
 
 p0 = r1 + beta*p0;
 
 r0 = r1;}
 }
 x = u0; \tcp{update value of the vector of solutions}
 
 \eIf{success = 0}{
 return -1;}{return niter;}
\end{algorithm}

\end{document}
