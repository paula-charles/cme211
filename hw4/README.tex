\documentclass{article}
\usepackage[utf8]{inputenc}

\title{HW4-README}
\author{Paula Charles}
\date{October 2021}

\begin{document}

\maketitle
\noindent
I constructed the \textbf{class Truss} to create a 2D truss
and to analyze it and find out its stability and forces. It
 takes as inputs a joints file and a beams file that are all
 we need to create the truss system.
\newline

\noindent
There is an optional 3rd parameter that provides the location
 of an output plot. If this location is precised, the
 \_\_init\_\_() method creates a plot that provides a 2D view 
of the truss and save it at that location. To do that, the 
\_\_init\_\_() method calls another method: 
\textbf{PlotGeometry()} which takes the location of the 
output plot as input and returns a file that is a plot of 
the 2D truss. This file is saved at the location asked.
\newline

\noindent
Overall, the \textbf{\_\_init\_\_()} method creates 
self.beams which is an array containing the beams and the 
joints they bring together and self.joints which is a 
dictionary containing the joints and all their 
characteristics.
\newline

\noindent
The matrix created to analyze the system has 2 rows per 
joint (one for the x equation, one for the y equation) 
and one column per beam and 2 additional columns per 
reaction force due to fixed support (one column for the 
x component, one for the y). We only need one column 
per beam force because we already know the direction of 
the beam force, so we only need to find out its norm. 
This is done in the \textbf{get\_matrix()} method. 
The resulting matrix is a sparse matrix: a CSR matrix.

\noindent
The method raises a Runtime error if the matrix is not
 square: the model will be overdetermined or 
underdetermined.

\noindent
This method does not return anything and does not 
take anything as input but uses variables to create 
a new attribute: self.matrix.
\newline

\noindent
The method \textbf{get\_vector()} creates a vector with 
the external forces to complete the equations (given 
that they will be on the RHS of the equation, we have 
put a minus sign before them). It enables us to have a 
full set of equation, that we can now solve. This 
method does not take inputs and creates self.vector.
\newline

\noindent
The method \textbf{solve\_matrix()} uses the matrix 
and vector created and solves the linear equations 
system. It uses SciPy to do so. If the system is 
singular, it will raise a Runtime error to signal that 
it can't solve the linear system. It returns a vector 
of values that have been found for the unknowns in our 
equations.
\newline

\noindent
Finally, the \textbf{\_\_repr\_\_} method creates a 
nice print. It calls the solve\_matrix() to get the 
values of the beam forces and displays them in a 
pleasing way.

\end{document}

